% ----------------------------------------------------------------------
\begin{frame}[c]{Formal Definition}{Stable models of positive programs}
  \begin{itemize}
  \item<2-> A set of atoms $X$ is \alert{closed under} a positive program $P$ iff
    \\for any $r\in P$, $\head{r}\in X$ whenever $\pbody{r}\subseteq X$
    \begin{itemize}
    \item $X$ corresponds to a model of $P$ (seen as a formula)
    \end{itemize}
    \medskip
  \item<3-> The \alert{smallest} set of atoms which is closed under a positive
    program $P$ is denoted by $\Cn{P}$
    \begin{itemize}
    \item $\Cn{P}$ corresponds to the $\subseteq$-smallest model of $P$ (ditto)
    \end{itemize}
    \medskip
  \item<4-> The set $\Cn{P}$ of atoms is the \alert{stable model} of a \emph{positive} program $P$
  \end{itemize}
\end{frame}
% ----------------------------------------------------------------------
\begin{frame}{Some ``logical'' remarks}
  \begin{itemize}
  \item <1-> Positive rules are also referred to as \alert<1-2>{definite clauses}
    \begin{itemize}
    \item Definite clauses are disjunctions with \alert<1>{exactly one} positive atom:
      \[
      a_0\vee\neg a_1\vee\dots\vee\neg a_m
      \]
    \item A set of definite clauses has a (unique) \alert<3>{smallest model}
    \end{itemize}
    \medskip
  \item<2-> \alert<2>{Horn clauses} are clauses with \alert<2>{at most} one positive atom
    \begin{itemize}
    \item Every definite clause is a Horn clause but not vice versa
    \item Non-definite Horn clauses can be regarded as integrity constraints
      \smallskip
    \item A set of Horn clauses has a \alert<3>{smallest model} or none
    \end{itemize}
    \medskip
  \item<3-> This \alert<3>{smallest model} is the intended semantics of such sets of clauses
    \begin{itemize}
    \item Given a positive program $P$,
    $\Cn{P}$ corresponds to the smallest model of the set of
    definite clauses corresponding to $P$
  \end{itemize}
\end{itemize}
\end{frame}
% ----------------------------------------------------------------------
\begin{frame}{Basic idea}
\pause
\begin{columns}
  \begin{column}<1-7>[t]{0.6\linewidth}
    Consider the logical formula \alert<2>{$\Phi$}
    and its three (classical) models:
    \visible<3>{%
    \begin{picture}(0,0)(-40,80)
      \alert{%
      \put(-90,50){\vector(2,-1){90}}
      \(
      \begin{array}{|rcl|}
        \hline
        p&\mapsto&1\\
        q&\mapsto&1\\
        r&\mapsto&0\\
        \hline
      \end{array}
      \)}
    \end{picture}}
    \[
    \alert<3>{\{p,q\}}, \{q,r\}, \text{ and }\{p,q,r\}
    \]
  \end{column}
  \begin{column}<1-7>[t]{0.35\linewidth}
    \(\arrayrulewidth=.1pt
    \begin{array}[t]{l|c|}
      \cline{2-2}
      \alert<2,4>{\Phi}&
      q
      \ \wedge \
      (q\wedge\neg r\to p)\\
      \cline{2-2}
    \end{array}
    \)
  \end{column}
\end{columns}

\bigskip

\uncover<5->{%
\begin{columns}
  \begin{column}<5-8>[t]{0.6\linewidth}
    Formula \alert<5>{$\Phi$} has one stable model,\\ often called \alert<5>{answer set}:
    \[
    \alert<5>{\{p,q\}}
    \]
  \end{column}
  \begin{column}[t]{0.35\linewidth}
  \uncover<6-8>{%
    \(\arrayrulewidth=.1pt
    \begin{array}[t]{l|rcl|}
      \cline{2-4}
      \alert<6>{P_\Phi}&
       q&\leftarrow&\\
      &\alert<8>{p}&\alert<8>{\leftarrow}& \alert<8>{q,\ \neg r}\\
      \cline{2-4}
    \end{array}
    \)}
\end{column}}
\end{columns}

\vfill

\uncover<7->{%
Informally, a set $X$ of atoms is a \alert{stable model} of a logic program $P$
\begin{itemize}
\item if $X$ is a (classical) model of $P$
  and
\item if all atoms in $X$ are \alert{justified} by some rule in $P$
\end{itemize}
% \pause
% {\small (rooted in intuitionistic logics HT~(Heyting,~1930) and G3~(G\"odel,~1932))}
}
\end{frame}
% ----------------------------------------------------------------------
\begin{frame}{Formal definition}{Stable models of normal programs}

  \begin{itemize}
  \item <1-> The \alert{reduct}, $\reduct{P}{X}$, of a program $P$ relative to
    a set $X$ of atoms is defined by
    \[
    \reduct{P}{X}
    =
    \{\head{r}\leftarrow\pbody{r} \mid r\in P \text{ and } \nbody{r}\cap X=\emptyset\}
    \]

  \item <2-> A set $X$ of atoms is a \alert{stable model} of a program $P$,
    if $\Cn{\reduct{P}{X}}=X$
    \bigskip
    \bigskip
  \item<3-> \structure{Remarks}
    \begin{itemize}
    \item<3-> $\Cn{\reduct{P}{X}}$ is the $\subseteq$--smallest (classical) model of \reduct{P}{X}
      \medskip
    \item<4-> Each atom in $X$ is justified by an \emph{``applying rule from $P$''}
    \item<4-> Set $X$ is \alert{stable} under \emph{``applying rules from $P$''}
    \end{itemize}
  \end{itemize}
\end{frame}
% ----------------------------------------------------------------------
\begin{frame}{A closer look at ${\reduct{P}{X}}$}

  \bigskip
  \begin{itemize}
  \item<1->
    Alternatively, given a set $X$ of atoms from $P$,

    \bigskip

    $\reduct{P}{X}$ is obtained from $P$ by \alert<1>{deleting}

    \medskip

    \begin{enumerate}\normalsize
    \item each \alert<1>{rule} having \alert<2>{\neg a} in its body with $a\in X$

      and then

      \smallskip

    \item all \alert<1>{negative atoms} of the form \alert<2>{$\neg a$}  \\
      in the bodies of the remaining rules
    \end{enumerate}
    \bigskip
  \item<2-> \structure{Note} \ Only \alert<2>{negative body literals} are evaluated
  \end{itemize}
\end{frame}
% ----------------------------------------------------------------------
\section{Examples}
% ----------------------------------------------------------------------
\begin{frame}{Example one}
\[
P
=
\left\{
    p \leftarrow p, \
    q \leftarrow \only<-8>{\neg p}\only<9>{\alert{\neg p}}
\right\}
\]
\pause
\newcommand{\PIip}{%
    \begin{array}[t]{lll}
      p &\leftarrow&p
      \\
       &&
    \end{array}}
\newcommand{\PIop}{%
    \begin{array}[t]{lll}
      p &\leftarrow&p
      \\
      q &\leftarrow&
    \end{array}}
\[
\begin{array}{c|cl|ccr}
       X                   &\ &\uncover<3-7>{P^X}&\ &\uncover<3-7>{\Cn{P^X}}&                              \\\hline
\{\phantom{p         ,q}\} &  &\uncover<3-7>{\PIop}&  &\uncover<3-7>{\{q\}}     &\uncover<4->{\KO}             \\\hline
\{         p\phantom{,q}\} &  &\uncover<3-7>{\PIip}&  &\uncover<3-7>{\emptyset} &  \only<5-8>{\KO}\only<9>{\OK}\\\hline
\{\phantom{p,}        q \} &  &\uncover<3-7>{\PIop}&  &\uncover<3-7>{\{q\}}     &\uncover<6->{\OK}             \\\hline
\{         p         ,q \} &  &\uncover<3-7>{\PIip}&  &\uncover<3-7>{\emptyset} &  \only<7-8>{\KO}\only<9>{\OK}
\end{array}
\]

\end{frame}
% ----------------------------------------------------------------------
\begin{frame}{Example two}
\[
P
=
\left\{
    p \leftarrow \only<-8>{\neg q}\only<9>{\alert{\neg q}}, \
    q \leftarrow \only<-8>{\neg p}\only<9>{\alert{\neg p}}
\right\}
\]
\pause
% \newcommand{\PI}{%
%     \begin{array}[t]{lll}
%       p &\leftarrow& \neg q
%       \\
%       q &\leftarrow& \neg p
%     \end{array}}
\newcommand{\PIipiq}{%
    \begin{array}[t]{lll}
      &&
      \\
      &&
    \end{array}}
\newcommand{\PIipoq}{%
    \begin{array}[t]{lll}
      p &\leftarrow&
      \\
      &&
    \end{array}}
\newcommand{\PIopiq}{%
    \begin{array}[t]{lll}
      &&
      \\
      q &\leftarrow&
    \end{array}}
\newcommand{\PIopoq}{%
    \begin{array}[t]{lll}
      p &\leftarrow&
      \\
      q &\leftarrow&
    \end{array}}
\[
\begin{array}{c|cl|ccr}
       X                   &\ &\uncover<3-7>{P^X}  &\ &\uncover<3-7>{\Cn{P^X}}&              \\\hline
\{\phantom{p         ,q}\} &  &\uncover<3-7>{\PIopoq}&  &\uncover<3-7>{\{p,q\}}   &\uncover<4->{\KO}\\\hline
\{         p\phantom{,q}\} &  &\uncover<3-7>{\PIipoq}&  &\uncover<3-7>{\{p\}}     &\uncover<5->{\OK}\\\hline
\{\phantom{p,}        q \} &  &\uncover<3-7>{\PIopiq}&  &\uncover<3-7>{\{q\}}     &\uncover<6->{\OK}\\\hline
\{         p         ,q \} &  &\uncover<3-7>{\PIipiq}&  &\uncover<3-7>{\emptyset} &  \only<7-8>{\KO}\only<9>{\OK}
\end{array}
\]
\end{frame}
% ----------------------------------------------------------------------
\begin{frame}{Example three}
\[
P
=
\left\{
    p \leftarrow \only<-6>{\neg p}\only<7>{\alert{\neg p}}
\right\}
\]
\pause\bigskip
% \newcommand{\PI}{%
%     \begin{array}[t]{lll}
%       p &\leftarrow& \neg p
%     \end{array}}
\newcommand{\PIip}{%
    \begin{array}[t]{l}
    \end{array}}
\newcommand{\PIop}{%
    \begin{array}[t]{l}
      p \leftarrow \
    \end{array}}
\[
\begin{array}{l|cl|ccr}
             X  &\ &\uncover<3-5>{P^X}&\ &\uncover<3-5>{\Cn{P^X}} &                  \\\hline
\{\phantom{p}\} &  &\uncover<3-5>{\PIop}&  &\uncover<3-5>{\{p\}}      &\uncover<4->{\KO} \\\hline
\{         p \} &  &\uncover<3-5>{\PIip}&  &\uncover<3-5>{\emptyset}  &  \only<5-6>{\KO}\only<7>{\OK}
\end{array}
\]
\bigskip
\end{frame}
% ----------------------------------------------------------------------
%
%%% Local Variables:
%%% mode: latex
%%% TeX-master: "../../main"
%%% End:
