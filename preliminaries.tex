% ----------------------------------------------------------------------
\begin{frame}<-9>{What is an interpretation?}
  \bigskip
  \begin{itemize}
  \item<2-> \structure{Assignment} \
    A function mapping variables to values
    \begin{itemize}
    \item<only@4-5> \structure{Example} \
      \(
      \ass: \{x,y,z\}\rightarrow\mathbb{N}
      \)
      such that
      \(
      \ass=\{x\mapsto 3,y\mapsto 1,z\mapsto 7\}
      \)
    \end{itemize}
    \smallskip
  \item<3-> \structure{Solution} \
    An assignment satisfying a set of constraints
    \begin{itemize}
    \item <only@5> \structure{Example} \
      \ass\ is a solution of
      \(
      \{2x<z, x+y<2z\}
      \)
    \end{itemize}
  \item[] \smallskip
  \item<only@7-> \structure{Interpretation} \
    An assignment mapping variables to truth values
    \begin{itemize}
    \item<only@8-12>
      Truth values \true\ and \false\ stand for true and false
      \only<8-12>{\par  (but there may be others)}
      \smallskip
    \item<only@9-12> \structure{Example} \
      \(
      B: \{a,b,c\}\rightarrow\{\true,\false\}
      \)
      such that
      \(
      B=\{a\mapsto\true, b\mapsto\true,c\mapsto\false\}
      \)
      % \smallskip
    % \item<only@10-12> An interpretation satisfies a formula if it evaluates the formula to \true
    \end{itemize}
    % \smallskip
  % \item<only@11-> \structure{Model}\
  %   An interpretation satisfying a set of formulas (or rules)
  %   \begin{itemize}
  %   \item <only@12-12> \structure{Example} \
  %     $B$ is a model of
  %     \(
  %     \{a\wedge b, a\vee c\}
  %     \)
  %   \end{itemize}
  % \item[] \smallskip
  % \item<only@14-> \structure{Representation}
  %   \par
  %   \smallskip
  %   We often denote interpretations by the set of their true atoms
  %   \begin{itemize}
  %   \item<only@15> \structure{Example} \
  %     We use
  %     \(
  %     \{a,b\}
  %     \)
  %     to represent
  %     \(
  %     \{a\mapsto\true, b\mapsto\true,c\mapsto\false\}
  %     \)
  %   \end{itemize}
  % \item <only@16>[]
  %   and (alternatively) use sets as the semantic cornerstones \par (rather than assignments or interpretations)
  \end{itemize}
\end{frame}
% ----------------------------------------------------------------------
%
%%% Local Variables:
%%% mode: latex
%%% TeX-master: "../../main"
%%% End:
