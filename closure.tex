% ----------------------------------------------------------------------
\begin{frame}{What is the meaning of a \alt<1>{logic program}{set~of\only<3->{~positive}~rules}?}
  \begin{itemize}
  \item<4-> \structure{Idea} \ The set\only<5->{ $X$} of atoms derivable from the rules in program\only<5->{ $P$}
  \item<6-> \structure{Question} \ How to characterize $X$ given $P$?
    \bigskip
  \item<only@7-27> \structure{Procedural characterization}
    \smallskip
  \item<only@8-27> [] \only<9->{\textbf{let} $\T{P}{X}=\{\head{r}\mid r\in P, \pbody{r}\subseteq X\}$ \textbf{in}} \
    \begin{itemize}\normalsize
    \item []$X := \emptyset$
      \smallskip
    \item []\textbf{while} \alt<9->{$\T{P}{X}\neq X$}{$\{\head{r}\mid r\in P, \pbody{r}\subseteq X\}\neq X$}
      \begin{itemize}\normalsize
      \item[] $X := \alt<9->{\,\T{P}{X}}{\{\head{r}\mid r\in P, \pbody{r}\subseteq X\,\}}$
      \end{itemize}
    \item[] \textbf{return} $X$
    \end{itemize}
    \item<only@10-26> \structure{Example} \
      $P=\{a\leftarrow{},b\leftarrow a\only<19->{,d\leftarrow c}\}$

      \begin{tabular}{cccc}
      \only<11-17,20-26>{\ $X_0=\emptyset$}&
      \only<13-17,22-26>{\ $X_1=\{a\}$}&
      \only<15-17,24-26>{\quad $X_2=\{a,b\}$}&
      \only<17,26>{\quad $X=\{a,b\}$}\\
      \only<12-17,21-26>{$\T{P}{X_0}=\{a\}$}&
      \only<14-17,23-26>{$\T{P}{X_1}=\{a,b\}$}&
      \only<16-17,25-26>{$\T{P}{X_2}=\{a,b\}$}&
      \end{tabular}
    \item<only@27> \structure{Procedural answer} \ The ``meaning of $P$'' is given by
      \par the value $X$ returned by the procedure applied to $P$

  \item<only@28-58> \structure{Mathematical characterization}
    \begin{itemize}\normalsize
    \item<only@29-58> A set $X$ of atoms is \alert{closed} under a positive program $P$ if
      \par\smallskip
      $\head{r}\in X$ whenever $\pbody{r}\subseteq X$ for all $r\in P$
      \smallskip
    \item<only@30-55> \structure{Example} \
      $P=\{a\leftarrow{},b\leftarrow a\only<39->{,d\leftarrow c}\}$
      \par\smallskip
      \only<31-38,40-55>{Is}
      \only<31-32,40-41>{$\emptyset$}%
      \only<33-34,42-43>{$\{a\}$}%
      \only<35-36,44-45>{$\{b\}$}%
      \only<37-38,46-47>{$\{a,b\}$}%
      \only<      48-49>{$\{a,b,c\}$}%
      \only<      50-51>{$\{a,b,d\}$}%
      \only<      52-53>{$\{a,b,c,d\}$}%
      \only<      54-55>{$\{a,b,c,d,e\}$}
      \only<31-38,40-55>{closed under $P$\,?}
      \only<32,41,34,43,36,45,49>{\KO}%
      \only<38,47,51,53,55>{\OK}

    \item<only@56-58> \structure{Mathematical answer} \ The ``meaning of $P$'' is given by
      \par the \alert{smallest} set of atoms closed under $P$
      \smallskip
    \item<only@57-58> and denoted by \Cn{P}\pause[58] and called the \alert{consequences} of $P$
    \end{itemize}
  \item<only@60-> \structure{Procedural answer} \ Value $X$ returned by the procedure applied to $P$
  \item<only@60-> \structure{Mathematical answer} \ Consequences $\Cn{P}$ of $P$
    \medskip
  \item<only@61-> \structure{Common answer} \ $X=\Cn{P}$
    \medskip
  \item<only@62-> \structure{Logical answer}
    \begin{itemize}\normalsize
    \item<63-> A set of atoms closed under $P$ is a model of $P$ and vice versa
      \smallskip
    \item<63-> \Cn{P} corresponds to the $\subseteq$-smallest model of $P$
    \end{itemize}
  \end{itemize}
\end{frame}
% ----------------------------------------------------------------------
%
%%% Local Variables:
%%% mode: latex
%%% TeX-master: "../../main"
%%% End:
