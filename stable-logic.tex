% ----------------------------------------------------------------------
\begin{frame}{Some ``logical'' remarks}
  \begin{itemize}
  \item <1-> Positive rules are also referred to as \alert<1-2>{definite clauses}
    \begin{itemize}
    \item Definite clauses are disjunctions with \alert<1>{exactly one} positive atom:
      \[
      a_0\vee\neg a_1\vee\dots\vee\neg a_m
      \]
    \item A set of definite clauses has a (unique) \alert<3>{smallest model}
    \end{itemize}
    \medskip
  \item<2-> \alert<2>{Horn clauses} are clauses with \alert<2>{at most} one positive atom
    \begin{itemize}
    \item Every definite clause is a Horn clause but not vice versa
    \item Non-definite Horn clauses can be regarded as integrity constraints
      \smallskip
    \item A set of Horn clauses has a \alert<3>{smallest model} or none
    \end{itemize}
    \medskip
  \item<3-> This \alert<3>{smallest model} is the intended semantics of such sets of clauses
    \begin{itemize}
    \item Given a positive program $P$,
    $\Cn{P}$ corresponds to the smallest model of the set of
    definite clauses corresponding to $P$
  \end{itemize}
\end{itemize}
\end{frame}
% ----------------------------------------------------------------------
%
%%% Local Variables:
%%% mode: latex
%%% TeX-master: "../../main"
%%% End:
