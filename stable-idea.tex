% ----------------------------------------------------------------------
\begin{frame}{What is a stable model?}
  \begin{itemize}
  \item<only@2-> \structure{Lessons learned} from positive programs
    \begin{itemize}\normalsize
    \item \Cn{P} is the smallest model of $P$, eliminating all other ones
    \item Every atom in \Cn{P} is justified by a proof
      \medskip
    \item<only@3-4> \structure{Example} \
      $P=\{a\leftarrow{},b\leftarrow a, d\leftarrow c\}$
      \ yields \
      $\{a,b\}$
      \ only
      \begin{itemize}\normalsize
      \item<4> $a$ \ is justified by \ $a\leftarrow$
      \item<4> $b$ \ is justified by \ $b\leftarrow a$ and $a\leftarrow{}$
      \end{itemize}
    \end{itemize}

  \item<only@6-11> \structure{Logical attempt}\only<11>{ \ \textcolor{red}{failed}}
    \begin{itemize}\normalsize
    \item<7-> \structure{Hypothesis} \ The smallest model of $P$\,?
    \item<only@8-> \structure{Example} \ $P=\{a\leftarrow\neg b\}$ \ yields \ $\{a\}$
    \item<only@9->[] but $P$ has \alt<10->{two minimal}{three} models, $\{a\}$\alt<10->{ and}{,} $\{b\}$
      \only<9>{, and $\{a,b\}$}
    \end{itemize}

  \item<only@12-> \structure{Procederal attempt\only<34>{ patched}} \only<33>{ \ \textcolor{red}{failed}}

  \item<only@34-> [] \textbf{guess} $X'$
  \item<only@13-> [] \textbf{let} $\T{P}{X}=\{\head{r}\mid r\in P, \pbody{r}\subseteq X\only<14->{,\nbody{r}\cap X\only<34->{'}=\emptyset}\}$ \textbf{in}
    \begin{itemize}\normalsize
    \item []$X := \emptyset$
      \smallskip
    \item []\textbf{while} {$\T{P}{X}\neq X$}
      \begin{itemize}\normalsize
      \item[] $X := {\,\T{P}{X}}$
      \end{itemize}
    \item[] \only<34->{\textbf{if} $X=X'$ \textbf{then} }\textbf{return} $X$\only<34->{ \textbf{else} \textbf{fail}}
    \end{itemize}
  \item<only@15-32> \structure{Example} \
    $P=\{a\leftarrow{},b\leftarrow a,\neg c\only<24->{,c\leftarrow b}\}$

    \begin{tabular}{ccccc}
      \only<16-22,25-32>{\ $X_0=\emptyset$}&
      \only<18-22,27-32>{\ $X_1=\{a\}$}&
      \only<20-22,29-32>{\quad $X_2=\{a,b\}$}&
      \only<      31-32>{\quad $X_2=\{a,b,c\}$}&
      \only<22         >{\quad $X=\{a,b\}$}\\
      \only<17-22,26-32>{$\T{P}{X_0}=\{a\}$}&
      \only<19-22,28-32>{$\T{P}{X_1}=\{a,b\}$}&
      \only<21-22,30-32>{$\T{P}{X_2}=\{a,b\only<30-32>{,c}\}$}&
      \only<      32-32>{$\T{P}{X_1}=\{a,b\}$}&
      \end{tabular}

  \end{itemize}
\end{frame}
% ----------------------------------------------------------------------
\begin{frame}{Basic idea}
\pause
\begin{columns}
  \begin{column}<1-7>[t]{0.6\linewidth}
    Consider the logical formula \alert<2>{$\Phi$}
    and its three (classical) models:
    \visible<3>{%
    \begin{picture}(0,0)(-40,80)
      \alert{%
      \put(-90,50){\vector(2,-1){90}}
      \(
      \begin{array}{|rcl|}
        \hline
        p&\mapsto&\true\\
        q&\mapsto&\true\\
        r&\mapsto&\false\\
        \hline
      \end{array}
      \)}
    \end{picture}}
    \[
    \alert<3>{\{p,q\}}, \{q,r\}, \text{ and }\{p,q,r\}
    \]
  \end{column}
  \begin{column}<1-7>[t]{0.35\linewidth}
    \(\arrayrulewidth=.1pt
    \begin{array}[t]{l|c|}
      \cline{2-2}
      \alert<2,4>{\Phi}&
      q
      \ \wedge \
      (q\wedge\neg r\to p)\\
      \cline{2-2}
    \end{array}
    \)
  \end{column}
\end{columns}

\bigskip

\uncover<5->{%
\begin{columns}
  \begin{column}<5-8>[t]{0.6\linewidth}
    Formula \alert<5>{$\Phi$} has one stable model,\\ often called \alert<5>{answer set}:
    \[
    \alert<5>{\{p,q\}}
    \]
  \end{column}
  \begin{column}[t]{0.35\linewidth}
  \uncover<6-8>{%
    \(\arrayrulewidth=.1pt
    \begin{array}[t]{l|rcl|}
      \cline{2-4}
      \alert<6>{P_\Phi}&
       q&\leftarrow&\\
      &\alert<8>{p}&\alert<8>{\leftarrow}& \alert<8>{q,\ \neg r}\\
      \cline{2-4}
    \end{array}
    \)}
\end{column}}
\end{columns}

\vfill

\uncover<7->{%
Informally, a set $X$ of atoms is a \alert{stable model} of a logic program $P$
\begin{itemize}
\item if $X$ is a (classical) model of $P$
  and
\item if all atoms in $X$ are \alert{justified} by some rule in $P$
\end{itemize}
% \pause
% {\small (rooted in intuitionistic logics HT~(Heyting,~1930) and G3~(G\"odel,~1932))}
}
\end{frame}
% ----------------------------------------------------------------------
%
%%% Local Variables:
%%% mode: latex
%%% TeX-master: "../../main"
%%% End:
